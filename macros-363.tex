%%  Common macros for the course CSC363H1F, Fall 2010,
%%  on the St. George Campus at the University of Toronto.
%%
%%  Copyright (c) 2009-2010 Francois Pitt <fpitt@cs.utoronto.ca>
%%  last updated at 17:01 (EST) on Mon 17 Jan 2011
%%
%%  This work may be distributed and/or modified under the conditions of
%%  the LaTeX Project Public License, either version 1.3 of this license
%%  or (at your option) any later version.
%%  The latest version of this license is in
%%      http://www.latex-project.org/lppl.txt
%%  and version 1.3 or later is part of all distributions of LaTeX
%%  version 2003/12/01 or later.
%%
%%  This work has the LPPL maintenance status "author-maintained".
%%
%%  The Current Maintainer of this work is
%%  Francois Pitt <fpitt@cs.utoronto.ca>.
%%
%%  This work consists of the file macros-363.tex.

% Trick so that math symbols are bold when text is.
\let\seiresfb\bfseries\def\bfseries{\boldmath\seiresfb}
\let\seiresdm\mdseries\def\mdseries{\unboldmath\seiresdm}

% Centered version of \llap and \rlap.
\newcommand*{\clap}[1]{\hbox to 0pt{\hss#1\hss}}

% Spacing macros with default argument.
\newcommand*{\vstretch}[1][1]{\vspace*{\stretch{#1}}}
\newcommand*{\hstretch}[1][1]{\hspace*{\stretch{#1}}}

% Fonts and characters.
\let\altemph\textsl
\let\strong\textbf
\let\code\texttt
\let\const\textsf
\let\latinabb\emph
\newcommand*{\txtbksl} {\symbol{"5C}}% \
\newcommand*{\txtcaret}{\symbol{"5E}}% ^
\newcommand*{\txtunder}{\symbol{"5F}}% _
\newcommand*{\txtlcurl}{\symbol{"7B}}% {
\newcommand*{\txtrcurl}{\symbol{"7D}}% }
\newcommand*{\txttilde}{\symbol{"7E}}% ~
\newcommand*{\N}{\mathbb{N}}  % requires amssymb
\newcommand*{\Z}{\mathbb{Z}}  % requires amssymb
\newcommand*{\Q}{\mathbb{Q}}  % requires amssymb
\newcommand*{\R}{\mathbb{R}}  % requires amssymb
\newcommand*{\bigOh}{\mathcal{O}}

% Abbreviations of latin phrases.
\newcommand*{\ie}{\latinabb{i.e.}}
\newcommand*{\eg}{\latinabb{e.g.}}
\newcommand*{\etc}{\latinabb{etc}}
\newcommand*{\vs}{\latinabb{vs}}

% Redefined symbols from amssymb.
\renewcommand*{\emptyset}{\varnothing}
\renewcommand*{\iff}{\mathrel{\Leftrightarrow}}
\renewcommand*{\implies}{\mathrel{\Rightarrow}}
\renewcommand*{\ge}{\geqslant}
\renewcommand*{\le}{\leqslant}

% For algorithms.
\let\proc\textsc
\let\kw\textbf
\let\var\textsl
%\newcommand*{\eq}{\mathrel{==}}
%\renewcommand*{\neq}{\mathrel{!\!=}}
%\renewcommand*{\leq}{\mathrel{<=}}
%\renewcommand*{\geq}{\mathrel{>=}}
%\renewcommand*{\gets}{\mathrel{=}}
%\newcommand*{\opgets}[1]{\mathrel{#1=}}
%\newcommand*{\True}{\const{True}}
%\newcommand*{\False}{\const{False}}
%\newcommand*{\None}{\const{None}}
%\newcommand*{\cmod}{\mathbin{\%}}

% Math macros.
\newcommand*{\comp}[1]{\overline{#1}}  % complement
\newcommand*{\emptystr}{\varepsilon}  % empty string
\newcommand*{\lequiv}{\mathrel{\ \Longleftrightarrow\ }}
\newcommand*{\lxor}{\oplus}
\newcommand*{\floor}[1]{\lfloor#1\rfloor}
\newcommand*{\ceil}[1]{\lceil#1\rceil}

% Like {itemize} but using less vertical space.
\newenvironment*{points}
 {\ifvmode\else\unskip\par\fi\begin{list}{$\bullet$}
   {\setlength{\topsep}{.25ex plus .125ex minus .1825ex}
    \setlength{\itemsep}{\topsep}\setlength{\parsep}{0ex}
    \setlength{\leftmargin}{1.75em}\setlength{\labelsep}{.5em}
    \setlength{\labelwidth}{1.75em}}\ignorespaces}
 {\ifvmode\else\unskip\fi\end{list}\ignorespaces}

% Checkboxes and checklists.
\newcommand*{\cbox}[1]{{\fboxrule.125ex\fboxsep0ex
    \framebox[2.75ex]{\rule[-.75ex]{0ex}{2.75ex}\smash{#1}}}}
\newcommand*{\cmark}{\raisebox{.0625ex}{$\surd$\kern.125ex}}
\newcommand*{\xmark}{\raisebox{-.1875ex}{\kern.0625ex{X}}}
\newcommand*{\checkbox}{{\sf\cbox{}}}
\newcommand*{\checkedbox}{{\sf\cbox{\cmark}}}
\newcommand*{\markedbox}{{\sf\cbox{\xmark}}}
\newenvironment{checklist}{\begin{list}{\checkbox}{}}{\end{list}}
\newcommand*{\checkeditem}{\item[\checkedbox]}
\newcommand*{\markeditem}{\item[\markedbox]}

% 363-specific: fonts and symbols.
\let\symb\mathord  % for TM symbols
\let\lang\textsc  % for languages
\let\class\textsl  % for complexity classes
\newcommand*{\B}{{\mathord\sqcup}}  % blank symbol for TM's
\newcommand*{\encode}[1]{{\left\langle#1\right\rangle}}  % for TM inputs
\newcommand*{\mapred}{\mathrel{\le_m}}  % mapping reducibility
\newcommand*{\polyred}{\mathrel{\le_p}}  % polytime reducibility
\newcommand*{\TM}{_{\textsc{tm}}}  % subscript "TM"
\newcommand*{\NTM}{_{\textsc{ntm}}}  % subscript "NTM"

% 363-specific: standard complexity classes.
\newcommand*{\co}{\class{co}}
\renewcommand*{\L}{\class{L}}
\renewcommand*{\P}{\class{P}}
\newcommand*{\NP}{\class{NP}}
\newcommand*{\PSPACE}{\class{PSPACE}}

% 363-specific: standard languages.
\newcommand*{\CNF}{\lang{CNF}}
\newcommand*{\SAT}{\lang{SAT}}
